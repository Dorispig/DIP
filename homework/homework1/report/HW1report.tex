\documentclass{article}
\usepackage[UTF8]{ctex}
\usepackage{amsmath}
\usepackage{amssymb}
\usepackage{tikz}
\usepackage{xcolor}
\usetikzlibrary{arrows,shapes,chains}
\usepackage{cite}
\usepackage{graphicx}
\usepackage{subfigure}
\usepackage{listings}
\usepackage{float}
%\usepackage[framed,numbered,autolinebreaks,useliterate]{mcode}

\title{HW1 实验报告}
\author{郑涛 SA24001077}
\date{\today}

\begin{document}
	\maketitle
	\section{问题描述}
	本次实验完成的任务是:
	\begin{enumerate}
		\item [(1)] 实现图像的简单变换,包括放缩、旋转、平移、翻转。
		\item [(2)] 用MLS和RBF算法实现图像扭曲。
	\end{enumerate}
	\section{程序思路说明}
	\subsection{图像变换}
	运用cv2.getRotationMatrix2D构建变换矩阵,加上平移参数,调用cv2.warpAffine即可完成变换。
	\subsection{图像扭曲}
	\subsubsection{MLS}
	本次实验实现的是论文\cite{MLS}中的Affine Deformations算法。
	
	MLS算法思路为最小化下方函数:
	\begin{equation}
		\sum_{i=0}^{n}w_i|l_v(x)-q_i|^2
	\end{equation}
	其中$p_i$为控制点,$q_i$为目标点,$w_i=\frac{1}{|p_i-v|^(2\alpha)}$。
	
	由于$l_v$为仿射变换,因此$l_v$可以写成$l_v(x)=xM+T$,$M$为变换矩阵,$T$为平移。论文中计算出了$T=q_*-p_*M$,其中
	$$p_*=\frac{\sum_{i=0}^{n}w_ip_i}{\sum_{i=0}^{n}w_i}$$
	$$q_*=\frac{\sum_{i=0}^{n}w_iq_i}{\sum_{i=0}^{n}w_i}$$
	$$\therefore l_v(x)=(x-p_*)M+q_*$$
	论文中提出的Affine\,Deformations算法:
	\begin{equation}
		M=(\sum_{i=0}^{n}\hat{p_i}^Tw_i\hat{p_i})^{-1}\sum_{j=0}^{n}w_j\hat{p_j}^T\hat{q_j}
	\end{equation}
	其中$\hat{p_i}=p_i-p_*$,$\hat{q_i}=q_i-q_*$。
	
	因此仿射变换可以写为(下面的$f_a$即为上文中的$l_v$):
	\begin{equation}
		f_a(v)=(v-p_*)(\sum_{i=0}^{n}\hat{p_i}^Tw_i\hat{p_i})^{-1}\sum_{j=0}^{n}w_j\hat{p_j}^T\hat{q_j}+q_*
	\end{equation}
	
	\subsubsection{RBF}
	本次实验实现的是论文\cite{RBF}中的算法。
	
	RBF算法思想是用径向基函数($Radial\,Basis\,Functions$)对像素点位置进行插值映射。
	插值函数:
	\begin{equation}
		\begin{aligned}
			T(x,y)=&(T_U(x,y),T_V(x,y))\\
			=&(\alpha_1+\alpha_2x+\alpha_3y+\sum_{i=1}^{N}a_ig(||x-x_i||),\\
			&\beta_1+\beta_2+\beta_3y+\sum_{i=1}^{N}b_ig(||y-y_i||))
		\end{aligned}
	\end{equation}
	由于条件个数少于未知数个数,在求解$T_U$(求解$T_V$时类似)时加上三个条件使得映射有唯一解:
	$$\sum_{i=1}^{N}a_iq(x_i)=0,for\quad q(x,y)=1,x,y$$
	
	本次实验选用的径向基函数是$g(r)=e^{-\frac{r^2}{\sigma^2}}$,该函数可以调整参数更好地实现局部映射。
	
	\section{编译环境}\noindent
	opencv-python==4.10.0.84\\
	numpy==1.23.5\\
	gradio==3.36.1
	\section{使用说明}
	run\_point\_transformer运行时可以选择MLS或者RBF,在代码第50,51行修改即可。
	\section{结果展示}
	\subsection{图像变换}
\begin{figure}[H]
	\centering
	\includegraphics[scale=0.4]{result/global/scale0_4}
	\caption{0.4scale}
	\label{fig:scale0_4}
\end{figure}
\begin{figure}[H]
	\centering
	\includegraphics[scale=0.4]{result/global/scale1_6}
	\caption{1.6scale}
	\label{fig:scale1_6}
	\end{figure}
\begin{figure}[H]
	\centering
	\includegraphics[scale=0.4]{result/global/rotate__66}
	\caption{-66rotate}
	\label{fig:rotate66}
\end{figure}
\begin{figure}[H]
	\centering
	\includegraphics[scale=0.4]{result/global/rotate_translationx}
	\caption{rotate+translationx}
	\label{fig:rotate_translationx}
\end{figure}
\begin{figure}[H]
	\centering
	\includegraphics[scale=0.4]{result/global/rotate_translationxy}
	\caption{rotate+translationy}
	\label{fig:rotate_translationxy}
\end{figure}

	\subsection{图像扭曲}
	\subsubsection{MLS}
\begin{figure}[H]
	\centering
	\includegraphics[scale=0.4]{result/point/MLS1}
	\caption{MLS1}
	\label{fig:mls1}
\end{figure}
\begin{figure}[H]
	\centering
	\includegraphics[scale=0.4]{result/point/MLS2}
	\caption{MLS2}
	\label{fig:mls2}
\end{figure}
\begin{figure}[H]
	\centering
	\includegraphics[scale=0.4]{result/point/MLS3}
	\caption{MLS3}
	\label{fig:mls3}
\end{figure}
\subsubsection{RBF}
\begin{figure}[H]
	\centering
	\includegraphics[scale=0.4]{result/point/RBF1}
	\caption{RBF1}
	\label{fig:rbf1}
\end{figure}
\begin{figure}[H]
	\centering
	\includegraphics[scale=0.4]{result/point/RBF2}
	\caption{RBF2}
	\label{fig:rbf2}
\end{figure}
\begin{figure}[H]
	\centering
	\includegraphics[scale=0.4]{result/point/RBF3}
	\caption{RBF3}
	\label{fig:rbf3}
\end{figure}
	
	\bibliographystyle{unsrt}
	\bibliography{bib/reference}
\end{document}